\subsection{Pseudocodigo y an\'alisis de complejidad}

Sea $_cursos$ un vector de tuplas donde se almacenan los cursos de entrada del problema y $_cursosRes$ el vector donde se almacenan las tuplas elegidas como soluci\'on al problema. En la primera coordenada de las tuplas se encuentra el incicio y en la segunda el fin de cada curso.

%%dejo mostrarcusos y verGraficamente???
\begin{algorithmic}
void resolver(){
	\item mostrar Cursos de entrada
	\item verlos Graficamente
	\item ordenar de menor a mayor _cursos de acuerdo a las segundas coordenadas de las tuplas
	\item int final=0;
	\For{int i=0;i<_cursos.size();++i}
	\If {_cursos[i].first.first > final}
		_cursosRes.push_back(_cursos[i]);
		final = _cursos[i].first.second;
	\EndIf
	\EndFor 
		
		
	

\end{algorithmic}

La funci\'on que se utiliza para ordenar es sort, y su complejidad es O(n.log(n)) con n el tamaño del vector a ordenar (http://www.cplusplus.com/reference/algorithm/sort/?kw=sort).

