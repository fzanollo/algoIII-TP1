\subsection{Tests}
\textbf{Experimentos con instancias aleatorias:}\\

Para generar estas instancias utilizamos la función rand() incluida en la Standard General Utilities Library\footnote{http://www.cplusplus.com/reference/cstdlib/}. Esta función genera números pseudo-random.\\

Nuestro generador de casos requiere como datos de entrada una cantidad máxima de paquetes, luego crea casos distintos desde 1 paquete hasta la cantidad pasada de la siguiente manera:\\
Genera un peso límite al azar entre 1 y 100.\\
Para cada paquete genera un peso individual al azar entre 1 y peso límite.\\

