\subsection{Pseudoc\'odigo}
\begin{algorithm}
\caption{Euclid’s algorithm}\label{euclid}
\begin{algorithmic}[1]
\Procedure{Euclid}{$a,b$}\Comment{The g.c.d. of a and b}
\State $r\gets a\bmod b$
\While{$r\not=0$}\Comment{We have the answer if r is 0}
\State $a\gets b$
\State $b\gets r$
\State $r\gets a\bmod b$
\EndWhile\label{euclidendwhile}
\State \textbf{return} $b$\Comment{The gcd is b}
\EndProcedure
\end{algorithmic}
\end{algorithm}

k camiones Resolver (Limite de peso, n, paquetes)\\
camiones = crear vector vacio()\\
heap camiones = crear heap vacio()\\
k=0\\
while queden paquetes\\

	tomo un paquete p\\
	
	if ( peso minimo(heap camiones) + p <= Limite de peso )\\
	
		le sumo el peso del paquete al camion que tenia el peso minimo del heap\\
		
		modifico el peso de ese camion en el vector camiones (accediendo mediante el
		 indice)\\
		 
	else\\
	
		k++ (agrego un camion)\\
		
		lo agrego al vector camiones\\
		
		lo agrego al heap\\
		
end while\\
return k+1 camiones\\
