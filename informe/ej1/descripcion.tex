\subsection{Descripci\'on}

% Describir detalladamente el problema a resolver dando ejemplos del mismo y sus soluciones.

El sistema utilizado por Pascal consiste en cargar paquetes, de acuerdo al orden de llegada, en camiones. Una vez que el primer cami\'on alcanza el peso m\'aximo, se llama un nuevo cami\'on para almacenar los paquetes que siguen.

El problema pide que se implemente el sistema utilizado por Pascal, con L (el peso m\'aximo soportado por todos los camiones), n (la cantidad de paquetes a cargar) y $p_1$, $p_2$, ..., $p_n$ (pesos de los paquetes) los par\'ametros de entrada.  Se supone que los paquetes no superan la capacidad m\'axima de los camiones.
Hagamos un ejemplo a modo de ilustraci\'on:

Sea L = 25, n = 7 y $p_1$ = 25, $p_2$ = 13, $p_3$ = 18, $p_4$ = 8, $p_5$ = 12, $p_6$ = 4 y $p_7$ = 1.
De acuerdo al m\'etodo del buen hombre, la soluci\'on es utilizar cuatro camiones con 25, 13, 18 y 25 sus pesos respectivos. El primer cami\'on contiene el primer paquete, el segundo cami\'on contiene el segundo paquete, el tercer cami\'on contiene el tercer paquete, el cuarto cami\'on contiene el cuarto, quinto, sexto y s\'eptimo paquete.

