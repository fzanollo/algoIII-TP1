\subsection{Descripci\'on}

% Describir detalladamente el problema a resolver dando ejemplos del mismo y sus soluciones.

El sistema utilizado por Pascual consiste en cargar paquetes en camiones, de acuerdo al orden de llegada. Cada paquete se intenta cargar en el cami\'on con menor peso (con peso mayor que cero), si no es posible se guarda en un nuevo cami\'on.

El problema pide que se implemente el sistema utilizado por Pascual, con L (el peso m\'aximo soportado por todos los camiones), n (la cantidad de paquetes a cargar) y $p_1$, $p_2$, ..., $p_n$ (pesos de los paquetes) los par\'ametros de entrada.  Se supone que los paquetes no superan la capacidad m\'axima de los camiones.
Hagamos un ejemplo a modo de ilustraci\'on:

Sea L = 25, n = 7 y $p_1$ = 25, $p_2$ = 13, $p_3$ = 18, $p_4$ = 8, $p_5$ = 12, $p_6$ = 4 y $p_7$ = 1.
De acuerdo al m\'etodo del buen hombre, la soluci\'on es utilizar cuatro camiones con 25, 21, 18 y 17 sus pesos respectivos. El primer cami\'on contiene el primer paquete, el segundo cami\'on contiene al segundo y al cuarto paquete, el tercer cami\'on contiene el tercer paquete, el cuarto cami\'on contiene al quinto, sexto y s\'eptimo paquete.

