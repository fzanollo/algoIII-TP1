\subsection{Complejidad}
Para programar este algoritmo utilizamos un par de contenedores de datos definidos en la Standard Template Library: vector y priority\_queue. \\
Las complejidades de las funciones de dichos contenedores fueron sacadas de sus respectivas documentaciones.\\

Para ver la documentación del contenedor vector:  http://www.cplusplus.com/reference/vector/vector/\\

Para ver la documentación del contenedor priority\_queue: 
http://www.cplusplus.com/reference/queue/priority\_queue/\\

Análisis de complejidad en el peor caso:\\

\textbf{Inicio del algoritmo}\\

Líneas 2 y 3: CrearVectorVacio()\footnote{http://www.cplusplus.com/reference/vector/vector/vector/}  y CrearHeapVacio()\footnote{http://www.cplusplus.com/reference/queue/priority\_queue/priority\_queue/}  tienen complejidad constante O(1).\\

Línea 4: Asignar un valor a una variable también O(1).\\

Línea 5 \textbf{Inicio del For}: Por cómo está definido el ciclo entra una vez por cada paquete. Complejidad lineal O(n), n=cantidad de paquetes.\\

Línea 6 \textbf{Guarda del If}: Obtener el mínimo del heap es O(1).\footnote{http://www.cplusplus.com/reference/queue/priority\_queue/top/} Luego la suma y la comparación de enteros son constantes. Complejidad total de la guarda: O(1) + O(1) = O(1)\\

Líneas 7 y 8 \textbf{Cuerpo del If}: Para poder modificar el elemento del heap debemos obtenerlo y luego reincorporarlo.\\
Es decir, llamamos una vez a la función 'pop' de priority\_queue (esta función es O(2.log(k)), k= cantidad de camiones hasta el momento) y una vez a la función 'push' de priority\_queue (esta función es O(log(k)) )\\
Luego debemos actualizar el peso del camión en el vector. Acceder al elemento y modificarlo es O(1).\\
Complejidad total del cuerpo del If: O(2.log(k)) + O(log(k)) + O(1) = O(3.log(k)) = O(log(k))\\

Líneas 11, 12 y 13 \textbf{Cuerpo del Else}: Modificar el valor de una variable es O(1).\\
Luego creamos un nuevo camión O(1), lo agregamos al vector de camiones O(1) y al heap (función 'push' de priority\_queue) O(log(k)).\\
Complejidad total del cuerpo del Else: O(1) + O(1) + O(log(k)) = O(log(k))\\

Línea 14 \textbf{Fín If}: Complejidad en el peor caso del If = O(log(k))\\

Línea 15 \textbf{Fín For}: Complejidad total: O(n).O(log(k)) = O(n.log(k))\\

\begin{flushleft}
\textbf{Complejidad total del algoritmo:} O(1) + O(1) + O(1) + O(n.log(k)) = O(n.log(k)) Siendo n=cantidad de paquetes y k=cantidad de camiones.\\
En el peor caso hay un camión por cada paquete, entonces a lo sumo k=n, si se quiere expresar la complejidad respecto sólo a los datos de entrada, una cota un poco exagerada (ya que el heap se va creando y sólo al final contiene los n camiones a la vez) sería \textbf{O(n.log(n))} Siendo n=cantidad de paquetes\\
\end{flushleft}







